\documentclass{article}
\usepackage[utf8]{inputenc}
\usepackage[T1]{fontenc}
\usepackage[french]{babel}
\usepackage{amsmath}
\usepackage{amssymb}

\begin{document}

\[
\boxed{
    \forall a, b, c \in \mathbb{Z}^* \quad [(a \mid bc) \land \text{GCD}(a,b)=1] \Rightarrow a \mid c
}
\]

\vspace{1cm} % Espace vertical

Par définition de la divisibilité :
\[ a \mid bc \Leftrightarrow \exists k \in \mathbb{Z} \text{  } ak = bc \]

Par le théorème de Bézout, puisque $\text{GCD}(a,b)=1$
\[ \exists u, v \in \mathbb{Z} \quad \text{  } a \cdot u + b \cdot v = 1 \]
\[ c(a \cdot u + b \cdot v) = c \]
\[ a \cdot u \cdot c + b \cdot c \cdot v = c \]

En substituant $bc$ par $ak$ (cf. ligne 1) :
\[ a \cdot u \cdot c + (ak) \cdot v = c \]

On factorise par $a$ :
\[ a (u \cdot c + k \cdot v) = c \]

Or, comme $u, c, k, v$ sont dans $\mathbb{Z}$, et que $c$ est non nul \[ (u \cdot c + k \cdot v) \in \mathbb{Z}^* \]

\[ (a \mid c) \quad \blacksquare \]

\end{document}